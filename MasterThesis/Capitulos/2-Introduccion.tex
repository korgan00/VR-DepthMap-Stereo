%----------------------------------------------------------------------------------------
%	INTRODUCCIÓN (2,3 hojas)
%----------------------------------------------------------------------------------------

\pagestyle{empty}
\chapter{Introducción}
% tema de trabajo
La realidad virtual es una tecnología que desde hace unos años ha estado intentando hacerse un hueco en la industria del entretenimiento.

Actualmente la realidad virtual tiene una variedad muy grande de dispositivos  con diferentes características. Las principales especificaciones a tener en cuenta a la hora de decidirnos uno en concreto son la disponibilidad de accesorios como los mandos, la resolución de pantalla y los grados de libertad tanto de las gafas como de los periféricos, siendo típicos 3 y 6 grados.

Los grados de libertad se definen como la capacidad de movimiento que permite un 

La realidad virtual es una tecnología que actualmente está en auge pero que sin embargo todavía esta construyendo una identidad propia. El contenido que se genera todavía esta basado en gran parte en técnicas ya conocidas como reproducción de vídeo y fotos cuya máxima adaptación consiste simplemente en poner una imagen ligeramente diferente en cada ojo.

Este trabajo trata de conseguir proporcionar a los usuarios de experiencias de Realidad Virtual mayor inmersión a la hora de ver contenidos que no están siendo generados en vivo, sino que han sido creados previamente ya sea con una cámara real o gráficos por ordenador.

La característica principal de este tipo de contenido es que cada imagen esta tomada desde un punto fijo en el espacio. Esto provoca una problemática que consiste en que la única libertad del usuario a la hora de visualizarlo en unas gafas de realidad virtual es el giro vertical y horizontal de la cabeza y provocando ver imágenes duplicadas si se inclina la cabeza hacia los hombros.
% interés
El principal motivo para realizar esta investigación es crear un método que permita dar más libertad con las gafas